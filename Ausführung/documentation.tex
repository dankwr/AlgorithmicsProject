\documentclass[a4paper, 12pt]{article}

\usepackage{amsmath, graphicx, url}
\usepackage{amsfonts}
\usepackage[T1]{fontenc}
\usepackage{comment}
\usepackage{tabto}


\title{Projekt: Die Nordischen Filmtage}
\author{
    Raik Dankworth}
\date{}

\begin{document}

\maketitle

\section{Einleitung}

In diesem Projekt geht es darum, sich ein Verfahren zu überlegen, mit dessen Hilfe man entscheiden kann, welche Filme man bei den Nordischen Filmtage besuchen sollte.
Die Anforderungen sind dabei, dass keine Filme doppelt besucht werden und man nur ein Film gleichzeitig ansehen kann.
Es ist der Plan zu wählen, mit dessen Hilfe man so viele gute Filme wie möglich sehen kann.
Um die Filme zu vergleichen, wird das ImDb-Ranking genutzt.
Dafür sind zu allen Filmvorstellungen angegeben, an welchen Tag und zu welcher Zeit welcher Film, identifiziert mit seinem Titel, gezeigt werden. 
Zusätzlich ist noch die Dauer der Vorstellung und das ImDb-Ranking des Films angegeben.
Die Laufzeit zwischen den Spielstätten spielt keine Rolle, d.h. am Ende einer Vorstellung kann direkt die nächste Vorstellung besucht werden.

\section{Vorgehen}

\subsection{Definition des ILP}
Das Schedule-Problem wird als 0-1-ILP modelliert.

\tabto{.03\linewidth} maximiere 
\begin{equation} \label{equ:objectFunction1}
    \sum_i^n r_i x_i
\end{equation}
\tabto{.03\linewidth} sodass 
\begin{equation} \label{equ:titleConstraints1}
    \forall f: \sum\limits_{i < n \> \land \> f = f_i} x_i \leq 1
\end{equation}
\begin{equation} \label{equ:overlappingConstraints1}
    \forall i < j < n \textmd{ mit  } t_i \cap t_j \neq \emptyset: x_i + x_j \leq 1
\end{equation}
\begin{equation} \label{equ:ilp1}
    \forall i < n: x_i \in \{0, 1\}
\end{equation}
Alle $n$ Filmvorstellungen werden durchnummeriert und mit den Variablen $x_i$ abgebildet.
Ist der Wert dieser Variablen 1, sollte diese Vorstellung besucht werden, sonst ist nur 0 zulässig \eqref{equ:ilp1}.
Es muss dabei jedoch auch beachtet werden, dass jeder Film $f$ nur einmal besucht werden darf.
Deshalb darf nur eine Filmvorstellung pro Film besucht werden \eqref{equ:titleConstraints1}.
Auch dürfen sich die Vorstellungszeiten $t_i$ der besuchten Vorstellungen nicht überschneiden \eqref{equ:overlappingConstraints1}.
Wir interessieren uns natürlich nur für die besten Filme, welche vorgestellt werden.
Dies wird über die Zielfunktion \eqref{equ:objectFunction1} mittels des Ranking $r_i$ abgebildet.

\subsection{Programmierung}

\section{Auswertung}


\end{document}
